\documentclass[9pt]{article}
\usepackage{amsmath,amssymb,graphicx}
\usepackage[spanish]{babel}
\usepackage[latin1]{inputenc}
\usepackage{vmargin}
\setpapersize{A4}
\setmargins{2.5cm}       
{1.5cm}                        
{16.5cm}                     
{23.42cm}                   
{10pt}                           
{1cm}                           
{0pt}                             
{2cm}                          

\author{Cristian Mauricio Borja}
\date{Jueves 25 de Mayo de 2017}
\title{\textbf{Resultados Tarea 5 - Metodos Computacionales}}

\begin{document}
\maketitle

\begin{abstract}
En este documento se presentan los resultados de la quinta y ultima tarea del curso Metodos Computacionales. El primer punto consisite en determinar el radio de un poro mientras que en el segundo punto se pide calcular los mejores parametros R y C, para ajustar una curva generada a partir de datos observacionales. Ambos ejercicios son desarrollados utilizando MCMC (Markov Chain Monte Carlo)
\end{abstract}

\section{Canales Ionicos}
En este ejercicio se busco determinar el circulo de radio maximo que puede ser dibujado en el interior de un canal ionico con el fin de estimar que tan grandes son estos poros. Mediante MCMC (Markov Chain Monte Carlo) se realizaron caminatas aleatorias para posibles coordenadas X y Y hasta dar con el punto que mejor se ajustase a los requerimientos. 

La logica del codigo fue partir de un punto Guess y encontrar la menor distancia entre dicho punto y cada molecula del canal ionico. La menor distancia implica el radio maximo posible en dicho punto. A partir de ahi las caminatas guardan el radio maximo asociado a cada punto y al final se indica en la grafica las coordenadas del punto cuyo radio es maximo.

El procedimiento fue realizado para ambos canales ionicos. A continuacion se presentan los resultados para cada Canal Ionico.


\subsection{Canal Ionico 1}

Los histogramas de frecuencias de aparicion de los valores de las variables durante las Caminatas Aleatorias fueron los siguentes:
	\begin{figure}[h]
		\centering
			\includegraphics[width=0.60\textwidth]{HX1.png}
		\caption{Histograma de frecuencias para los valores de la variable X (Canal1).}
		\label{fig:HX1}
	\end{figure}

	\begin{figure}[h]
		\centering
			\includegraphics[width=0.60\textwidth]{HY1.png}
		\caption{Histograma de frecuencias para los valores de la variable Y (Canal1).}
		\label{fig:HY1}
	\end{figure}
El circulo de Radio maximo hallado para el Canal Ionico 1, fue:
	\begin{figure}[h]
		\centering
			\includegraphics[width=0.60\textwidth]{RMC1.png}
		\caption{Circulo de Radio Maximo para el Canal 1.}
		\label{fig:RMC1}
	\end{figure}


\subsection{Canal Ionico 2}
Los histogramas de frecuencias de aparicion de los valores de las variables durante las Caminatas Aleatorias fueron los siguentes:
	\begin{figure}[h]
		\centering
			\includegraphics[width=0.60\textwidth]{HX2.png}
		\caption{Histograma de frecuencias para los valores de la variable X (Canal2).}
		\label{fig:HX2}
	\end{figure}

	\begin{figure}
		\centering
			\includegraphics[width=0.60\textwidth]{HY2.png}
		\caption{Histograma de frecuencias para los valores de la variable Y (Canal2).}
		\label{fig:HY2}
	\end{figure}

El circulo de Radio maximo hallado para el Canal Ionico 2, fue:
	\begin{figure}[h]
		\centering
			\includegraphics[width=0.60\textwidth]{RMC2.png}
		\caption{Circulo de Radio Maximo para el Canal 2.}
		\label{fig:RMC2}
	\end{figure}


\section{Circuito RC}
El modelo del que se parte esta dado en el enunciado, es la ecuacion que describe la carga de un condensador 
$$ 
q(t) = Q_{max}(1-e^{-t/RC}) 
$$
El Metodo de busqueda consistio de partir de un valor aleatorio para R y C (guess, en este caso ambos fueron 0). Y a partir de alli mediante Caminatas Aleatorias, la funcion LikeliHood se encargo de determinar los mejores parametros, a continuacion son presentadas las graficas obtenidas.

Las graficas de Verosimilitud para los valores de Resistencia y Capacitancia son respectivamente:
	\begin{figure}
		\centering
			\includegraphics[width=0.60\textwidth]{VSR.png}
		\caption{Grafica de Verosimilitud para los valores de Resistencia.}
		\label{fig:VSR}
	\end{figure}

	\begin{figure}
		\centering
			\includegraphics[width=0.60\textwidth]{VSC.png}
		\caption{Grafica de Verosimilitud para los valores de Capacitancia.}
		\label{fig:VSC}
	\end{figure}

Los Histogramas de Frecuencias para la aparicion de valores de R y C por las caminatas aleatorias son respectivamente:
	\begin{figure}
		\centering
			\includegraphics[width=0.60\textwidth]{HR.png}
		\caption{Histograma de frecuencias para los valores de Resistecia.}
		\label{fig:HR}
	\end{figure}

	\begin{figure}
		\centering
			\includegraphics[width=0.60\textwidth]{HC.png}
		\caption{Histograma de frecuencias para los valores de Capacitancia.}
		\label{fig:HC}
	\end{figure}

La grafica final, revela el mejor ajuste logrado, cuyos parametros R y C son mostrados en la misma.
	\begin{figure}
		\centering
			\includegraphics[width=0.60\textwidth]{MA.png}
		\caption{Grafica de los datos y el mejor ajuste dado por los parametros calculados.}
		\label{fig:MA}
	\end{figure}

\end{document}
